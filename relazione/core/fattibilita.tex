\section{Studio di fattibilità}

\subsection{Contesto d'uso}
Dalle analisi precedenti di user research e valutazione di sistemi
esitenti si può fare un quadro del contesto d'uso dell'applicazione
proposta.
Vedremo in particolare qual'è il target d'utenza, quali sono i task da
prendere in considerazione e quali sono i possibili vincoli tecnici e
ambientali.

\subsubsection{Target d'utenza}
Come emerge dalla user research il target d'utenza previsto è molto
eterogeneo, infatti non si sono riscrontrati dei ristrettivi limiti di
età, istruzione o competenze culinarie.\\
Forse l'unica vera restrizione riguarda la nazionalità degli utenti
in quanto si è deciso di concentrare
l'attenzione verso l'utenza italiana. Questa scelta è nata dall'analisi
delle statistiche le quali suggeriscono che l'Italia è una delle
prime nazioni nel mondo per quanto riguarda la passione culinaria
\ref{fig:cooking-country}. A conoscenza di ciò, focalizzarsi su un
utenza internazionale avrebbe potuto confondere gli italiani, poco familiari
con lo stile culinario non tradizionale. Ad esempio la colazione
italiana prevede un pasto molto rapido a differenza di molte colazioni
estere e la suddivisione delle portate in ``primo'' e ``secondo'', le quali
identificano già la categoria della pietanza, non è molto comune
all'estero.

\subsubsection{Task}
A seguito delle valutazioni dei sistemi esistenti, è risultato opportuno
rivisitare la lista dei task già definita nello user research
\ref{tasks}, con
l'idea di far forza sul contesto nel quale le applicazioni esistenti
mostrano le loro debolezze.
TODO modificare lista task:

\begin{enumerate}
\label{newtasks}
\item Ricerca di una ricetta per nome.
\item Ricerca di una ricetta per ingredienti.
\item Filtro di una ricetta per categoria.
\item Filtro di una ricetta in base alle intolleranze/allergie.
\item Suggerimenti sulle ricette di stagione.
\item Salvare una ricetta nei preferiti.
\item Rimuovere una ricetta dai preferiti.
\item Distinguere le varie fasi di una preparazione della ricetta.
\item Visualizzare le foto inerenti alla preparazione di
una ricetta.
\item Avanzare con le fasi di preparazione mediante messaggi vocali.
\item Text-to-Speech delle fasi di preparazione di una ricetta.
\item Individuare gli ingredienti necessari alla preparazione di una
ricetta.
\item Individuare la difficoltà di preparazione di una ricetta.
\item Inserire una nuova ricetta nel catalogo dell'applicazione.
\item Condividere una ricetta sui social network.
\item Commentare una ricetta in un apposito topic online.
\item Accedere alla lista della spesa.
\item Inserire gli ingredienti nella lista della spesa.
\item Rimuovere gli ingredienti dalla lista della spesa.
\item Modificare la grandezza del font di una ricetta.
\item Creare un menù completo selezionando una lista di ricette.
\item Selezionare la lingua dell'applicazione.
\end{enumerate}

\subsubsection{Vincoli tecnici ed ambientali}

\subsection{Scenari d'uso}

\subsection{Personas}
