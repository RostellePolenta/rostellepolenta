\section{Proposta di intervento}
\subsection{Introduzione}
In questo documento vengono spiegate le scelte progettuali che hanno portato alla realizzazione del prototipo di interfaccia mobile dell’applicazione denominata CookApp.\\
Inizialmente vengono descritte le scelte implementative dell'interfaccia adottate, identificando il design adatto.\\
Si continua con l'introduzione della \textit{Gamification} utile per un apprendimento veloce dell'applicazione e alla descrizione delle sfide che permettono di invogliare l'utente all'utilizzo costante.\\
Dopo un'esplicativa parte sull'archittettura dell'informazione, verrà proposto lo schema Blueprint dell'applicazione al fine di individuare tutti i passi che l'utente può fare durante l'esperienza di utilizzo.

\subsection{Scelte implementative}
Il modello scelto per l'applicazione segue l'approccio Goal-oriented, dato che il fine ultimo è il raggiungimento di un particolare obiettivo da parte dell'utente in un periodo breve/medio. L'obiettivo è quello di realizzare un piatto grazie all'aiuto fornito da CookApp o la creazione di un menù a partire dal vasto catalogo di ricette fornito.\\
Per implementare un design comodo e naturale all'utente medio, siamo partiti dagli esempi di applicazioni esistenti e abbiamo cercato i pregi di entrambe, verificando trasversalmente quali fossero i task più complessi rilevati duranti i test e riproponendoli in modalità più semplici.\\
Le scelte implementative si focalizzano quindi sulla facilità d'uso dell'applicazione, in modo che questi risulti di facile apprendimento fin da subito e che permetta all'utente medio di utilizzarla senza ricorrere all'help fornito ogniqualvolta non si ricordi una funzione.\\
Grazie all'analisi etnografica abbiamo constatato come l'utenza media sia molto eterogenea nel campo tecnologico e gastronomico; l'interfaccia è quindi studiata per venire incontro a tutte le possibili necessità diverse che possono avere utenti con esperienze passate diverse: si è utilizzato un linguaggio neutro, nè troppo specialistico nè troppo dettagliato, in modo che l'applicazione sia contemporaneamente discreta e non invasiva all'utente più preparato, che preferisce la velocità d'uso ad eventuali consigli.\\
Tecnicamente, cioè stato possibile implementando diverse viste nelle sezioni, come la possibilità di visualizzare una ricetta in 3 diverse modalità descritta successivamente o dando all'utente professionista la possibilità di scegliere se interagire con l'aspetto social o no.

\subsection{Gamification}
Per migliorare il coinvolgimento dell'utente nell'applicazione e per renderla più conforme agli applicativi moderni, si è deciso di affiancare alla parte più tradizionale di ricette e menù, una parte sociale.\\
Gli utenti, quindi, hanno la possibilità di interagire tra loro e collaborare.\\
Al fine di istruire l'utente inesperto e introdurre un livello di sfida nell'utente medio, si è scelto si utilizzare la gamification:
l'utente che crea un nuovo profilo, sarà inizialmente introdotto all'applicazione per mezzo di tutorial esplicativi e gli verrà assegnato un livello di esperienza di base.\\
Man mano che un utente acquisisce livelli di esperienza, espressi mediante numeri e nomi accattivanti, vengono sbloccate nuove funzioni, come la possibilità di aiutare utenti che fanno richieste live, creare discussioni nella sezione community o creare menù pubblici.\\
Solamente mediante il costante utilizzo delle altre funzioni disponibili, insieme al feedback positivo degli altri utenti, è possibile aumentare il proprio livello di esperienza, arrivando al completo dominio dell'applicazione.\\
L'utente professionista può sempre decidere di disattivare le funzioni social al fine di utilizzare l'applicazione come puro strumento d'aiuto durante la preparazione dei pasti, rinunciando però ad eventuali suggerimenti, spesso utili.\\
Questo approccio è stato pensato principalmente per soddisfare ogni tipologia di utente, da quella più inesperta, a quella più capace nell'ambito tecnologico e/o culinario, permettendo a entrambe le categorie un uso stimolante e mai noiose dell'applicativo.\\
L’applicazione infatti tiene in forte considerazione quelle persone che amano poter condividere le proprie realizzazioni in un contesto sociale adeguato, potendo chiedere aiuto a una community di utenti sempre pronta a dare suggerimenti e consigli, imparando ed aiutando allo stesso tempo.


\subsection{Architettura delle informazioni}


% -----------------------------------------------------------

\subsection{Design}
\textbf{Homepage}
Una volta effettuato il login, all'utente viene mostrata la homepage.\\
Subito si può notare la struttura generale dell'applicazione:
\begin{itemize}
\item Una barra orizzontale posta nella parte superiore dell'app permette di fare operazioni di undo/redo per tornare in schede già visitate in precedente o per annullare certe azioni; al centro viene sempre visualizzato il logo appositamente creato mentre a destra è possibile cercare una ricetta o un menù semplicemente immettendo il nome della ricetta stessa o degli ingredienti separati da virgole; infine l'avatar a destra permette di accedere al proprio profilo.
\item La barra verticale posta a sinistra è il menù delle sezioni dell'applicazione; in ordine troviamo la homepage, il ricettario, i menù, la lista della spesa, la sezione community, le impostazioni e l'help. In fondo è visibile l'icona per i comandi vocali attivabili tramite il comando "Ehy CookAp".
\item In basso a destra è sempre visibile un'icona raffigurando la chat delle richieste di aiuto live attive. Nel momento in cui si riceve una risposta a qualche richiesta, l'icona cambia e diventa di colore giallo, in attesa di una sua lettura. Successivamente viene descritta più approfonditamente la chat.
\item Infine al centro della pagina è visibile il contenuto iniziale della homepage, in questo caso.
\end{itemize}
La homepage è il punto di ingresso dell'applicazione: al centro è possibile avere un'overview generale, un \textit{Cover flow} sfogliabile con swipe orizzontali, permette la visualizzazioni del piatto del giorno, del menù del giorno e di piatti legati a festività o eventi inerenti al periodi.\\
Sotto, invece, vi sono due scorciatoie per le ricette e i menù recentemente visitati, insieme a una serie di riquadri delle ricette più cliccate in generale.

\textbf{Aiuto LIVE}
Ogniqualvolta un utente chieda aiuto alla community durante la preparazione di un piatto, viene automaticamente creata un'apposita voce nella sezione, sempre consultabile, degli aiuti live.\\
Ordinati in ordine cronologico, ogni voce è eliminabile in ogni momento e si presenta di diversi colori:
\begin{itemize}
\item Bianco: la richiesta è presente, ma non vi sono ancora risposte
\item Giallo: la richiesta ha alcune risposte, ma nessuna è stata selezionata come esaustiva
\item Verde: la richiesta è stata soddisfatta e non è più presente nelle richieste pendenti.
\end{itemize}

\textbf{Ricettario}
Il ricettario è il cuore dell'intero catalogo di CookApp: svariate categorie sono presentate all'utente disposte in un design a mosaico. Le categorie più gettonate tendono a coprire un'area maggiore rispetto alle ricette meno selezionate dalla comunità.\\
Con un semplice tocco delle dita è possibile aprire ogni categoria per ottenere la lista delle ricette.\\
L'utente è inoltre invitato a inserire una propria ricetta mediante l'apposito pulsante oppure può filtrare i risultati visibili cliccando sul simbolo dell'imbuto.\\\\

Ogni categoria mostra le ricette disposte in ordine alfabetico (se l'ordinamento è attivo) ed è possibile già individuare eventuali ricette vegetariane o vegane dall'apposito simbolo presente nei riquadri.\\
L'apposito filtro permette di selezionare determinate tipologie di ricette, in base agli ingredienti presenti, facendo anche particolarmente attenzione a diete particolari e all'intolleranza al lattosio e al glutine. Ulteriori parametri sui quali è possibile fare l'operazioni di filtro sono il tempo il preparazione e le variante vegatariane e vegane di piatti simili.\\
L'opzione "Cos'hai in frigo" è appositamente studiata in CookApp per collegarsi con i modelli più moderni di smartFridge: la sincronizzazione cloud con il frigorifero permetterà di filtrare i risultati in base ai prodotti freschi presenti nel frigo. L'opzione è ovviamente disabilitata se non è presente uno smartFridge collegato.\\

\textbf{Ricetta}
Ogni ricetta presenta una lunga schermata visualizzabile mediante lo swipe verticale della pagina.\\
Il banner iniziale mostra la fotografia di copertina della ricetta scelta dall'utente, insieme alla votazione media delle recensioni espressa su 5 stelle e alla possibilità di eliminare la ricetta dal catalogo se si è l'autore.\\
Tempo di preparazione, di cottura, difficoltà di realizzazione e costo delle materie prime, sono ben visibili affianco alla galleria di fotografie della ricette; sempre qui è inoltre possibile sapere quanti utenti hanno aggiunto la ricetta ai propri preferiti.\\
La sezione successiva mostra le varie quantità di ingredienti necessari, in sistema metrico (se non diversamente richiesto nelle impostazioni) e permette di adeguare ogni quantità in base alle persone richieste. Tramite l'apposito pulsante è possibile aggiungere tutti gli ingredienti, automaticamente, nella lista della spesa, raggiungibile cliccando sul quarto simbolo a sinistra.\\
La fase di preparazione mostra 3 diverse tipologie di vista, per soddisfare ogni tipologia di utente: da quello che necessita più dettagli e aiuti nella preparazione del piatto, al professionista.\\
\begin{enumerate}
\item La vista "Passo-passo 3D" è una nuova funzionalità esclusiva che permette di interagire a 360 gradi con una riproduzione 3D del piatto che si sta preparando. Ogni fase è accompagnata da un'animazione in tre dimensioni, che permette di verificare la corretta procedura su più punti di visione.
\item La vista "Passo-passo video" affianca la descrizione testuale della ricetta alla visione di un video che mostra la procedura tramite il player. Il sistema è ottimizzato per sincronizzare il tempo di riproduzione del video e la fase del procedimento corrispondente.
\item La "Vista compatta" è quella classica, dove vengono mostrate tutte le fasi di preparazione, lasciando all'utente il compito di sincronizzare lavoro e testo scritto.
\end{enumerate}
Ogni step di preparazione mostra sempre la possibilità di chiedere alla community, aiuto riguardo quel preciso passaggio. Automaticamente verrà creata una nuova voce della sezione "Aiuti live" già descritta in precedenza.\\
Nelle ultime sezioni vi sono i pulsanti di interazione con la ricetta: è qui possibile aggiungere la ricetta ai preferiti, aggiungere la ricetta ad un menù personale (con la possibilità di crearne uno direttamente da qui) e la possibilità di condividere la ricetta sui social network o sui servizi di messaggistica.\\
Infine la sezione "Recensioni" permette sia la scrittura che la lettura delle opinioni di altri utenti sulla ricetta.\\

\textbf{Aggiunta di una ricetta}
Mediante all'apposito pulsante "Aggiungi una ricetta" è possibile arricchire il catalogo di CookApp.\\
Si interagisce quindi con una serie di fasi che guidano passo-passo l'utente all'inserimento di tutte le informazioni necessarie: qui vi si trova la possibilità di inserire tutte i dati visti in precedenza nella sezione "Ricette".\\
Si noti come, nella fase 3, è possibile abbinare a ogni descrizione di fase di preparazione, una fotografia esplicativa o un video. Sempre in questa sezione, il tasto "Aggiungi un'altra fase" permette di inserire qualsiasi numero di step necessari al completamento della ricetta.\\
L'ultima fase, ovvero la quarta, permette all'utente di indicare se il piatto è vegetariano e se è rappresenta una variante a un piatto simile, ma contenente carne. Stessa procedura è prevista se è un piatto vegano.\\
Dopo aver impostato difficoltà, costo e numero di persone, si invita l'utente a inserire un set di tag per classificare correttamente la ricetta: se il tag inserito è riconosciuto dal sistema, verrà visualizzato in un riquadro colorato.\\
Completato l'inserimento di tutti i dati richieste l'utente non deve fare altro che confermare l'inserimento della ricetta.\\
Il sistema conclude l'operazione assegnando un quantitativo di punti esperienza permettendo all'utente di avanzare di livello (vedi sez. Gamification).\\

\textbf{Profilo}
La sezione profilo mostra inizialmente alcune delle informazioni inserite nella fase di registrazione.\\
Nella sezione "Generale" è inizialmente mostrato un riepilogo delle azioni svolte di recente dall'utente o da altri utenti se questi hanno interagito con ricette/menù pubblicati. Sotto invece è possibile tenere d'occhio i progressi ottenuti nel sistema mediante un indicatore di punti esperienza e dei livelli sbloccati e sbloccabili.\\
Le sezioni "Le mie ricette" e "I miei menù", come suggerisce il nome indica ricette e menù pubblicati nel sistema con una rapida visione del numero di utenti che hanno inserito le ricette nei propri preferiti.\\
I propri preferiti sono archiviati nella sezione apposita: da qui è possibile rimuoverli, semplicemente trascinandoli verso il basso, verso la barra colorata con il simbolo del cestino.\\
Nel pannello di modifica del profilo è possibile modificare molti dei parametri inseriti durante la registrazione, insieme al livello di esperienza:
\begin{itemize}
\item L'utente novizio è quell'utente che non ha esperienza nè in ambito culinario nè in ambito tecnologico; il sistema è progettato per seguire passo-passo l'utente, offrendogli spesso messaggi di aiuto nelle operazioni svolte, come farebbe un amico fidato.
\item L'utente medio è l'utente che sa interagire bene col sistema, ma preferisce mantenere attive le funzionalità social per scambiare opinioni con la comunità ed eventualmente chiedere aiuto durante la preparazione della ricetta.
\item L'utente esperto non ha bisogno di funzionalità social che interferiscono con un uso metodico del sistema. 
\end{itemize}

\textbf{Impostazioni}
Qui vi sono le principali impostazioni previste dall'applicazione.\\ Normalmente questa imposterà di default la lingua del sistema, ma se si vuole è possibile cambiarla da un apposito menù a tendina.\\
Le difficoltà della lettura possono essere risolte aumentando qui la grande del font.\\
Se i suoni dell'applicazione o delle notifiche provoca disturbo, qui è possibile disattivarle.\\
Infine è possibile definire qual è la visualizzazione preferita dei procedimenti delle ricette, insieme al sistema di misura utilizzato per pesare gli ingredienti.\\

\textbf{Help}
In questa sezione l'utente troverà, in ogni momento, la possibilità di leggere in modo dettagliato ogni funzionalità dell'applicazione, insieme alle risposte delle domande più frequenti (FAQ).
