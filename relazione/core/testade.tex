**DRAFT TEST
* giallo zafferano
adelaide
età: 21

\item Ricerca di una ricetta per nome.
giallo zafferano
Vado sulla barra cerca ricette perchè abbastanza intuitivo e scrivo
"pasta" e durante l'inserimento testo noto che vengono
suggerite delle keyword di ricerca. Quest'ultime mi confondono perchè
sembrano selezionabili solo le keyword ma non una ricerca specifica. Ritento
scrivendo "pasta carbonara".
Si è apre una schermata con una
lista di ricette disponbili. Vengo distratto da ricette che non mi
interessano. I criteri di ricerca sembrano poco precisi.

all the cooks
Cerco una barra di ricerca dalla "home" e non la trovo. Insulto gli
sviluppatori dell'applicazione.
Solo dopo aver effettuato il task di filtro per categorie ho notato la
presenza della barra di ricerca.


\item Ricerca di una ricetta per ingredienti.
giallo zafferano
Vado nella home. Mi aspetto che ci sia un riquadro per selezionare un
tot di ingredienti e avviare una ricerca. 

all the cooks
Vado nella home. Mi aspetto che ci sia un riquadro per selezionare un
tot di ingredienti e avviare una ricerca. 

\item Filtrare la lista di ricette per categoria.
giallo zafferano
Vado nella barra di ricerca, scrivo "Carne", clicco sulla keyword
"Carne" suggerrita nel menu a comparsa. Ottengo una lista di ricette ma
avrei preferito una lista di sottocategorie perchè confonde.

all the cooks
Individuo subito la sezione delle categorie. Selezionando quest'ultima
noto la lista dei filtri selezionabili. Spunto la categoria "dessert"
dalla lista dei filtri e compare una lunga lista di dessert. Cercando
una lista di sottocategoria, seleziono il menu a tendina "Categories" ma
l'applicazione si chiude.

\item Filtrare la lista di ricette in base alle intolleranze/allergie.
giallo zafferrano
Mi aspetterei di trovare una categoria apposita "intolleranze". Non la
trovo e provo ad effettuare una ricerca con il termine "intelleranza".
Ottengo una lista di sole  5 ricette. Quindi una lista piuttosto scarna
quindi mi ritengo piuttosto insoddisfatta.

all the cooks
Si può filtrare la lista delle ricette tramite il menu a tendina
"Dietary Preferences" potendo selezionare diversi tipi di dieta tra cui
specifiche per intolleranze. Soddisfatta. 

\item Visualizzare ricette di stagione.
giallo zafferrano
Vado nella home, seleziono categorie, poi la sottocategorie "feste" poi
si apre la schermata con le varie occorrenze. Il periodo corrente non è
suggerrito ma selezionabile da una lista.

all the cooks
Noto la sezione Trending che mi sembra oppurtuna. Ma dalla lista delle
ricette mi sembra una categoria non molto coerente con il nome.

\item Salvare una ricetta nei preferiti.
giallo zafferano
Dalla schermata di consultazione di una ricetta. Deduco che non si
può aggiungere la ricetta alla lista dei preferiti.

all the cooks
Non trovo un pulsante apposito nella visualizzazione di una ricetta
specifica. Lo trovo però nella home page per la ricette in primo piano.
Clicco sul pulsante Favorites e compare un form. Non capisco se la
ricetta è stata aggiunta o meno ai preferiti. Non capisco cosa vuol dire
la input box Tag.

\item Accedere alla lista delle ricette preferite.
giallo zafferrano
Ho cercato il pulsato apposito nella schermata home ma non l'ho trovato.

all the cooks
non trovando pulsanti idonei mi butto a premere l'unico pulsante
inerente al task ovvero quello per aggiungere una ricetta in primo piano
ai preferiti. Non riesco comunque a farlo.

\item Rimuovere una ricetta dai preferiti.
Fail.

\item Distinguere le varie fasi di una preparazione della ricetta
selezionata.
giallo zafferrano
Individuo subito la spiegazione delle varie fasi di preparazione, ognuna
numerata e accompagnata da una foto esplicativa. Direi che è intuitivo e
ben spiegato.

all the cooks
Individuo subito le fasi di preparazione ma mi aspetterei le foto
corrispondenti che invece non vengono visualizzate.

\item Visualizzare le foto inerenti alla preparazione di
una ricetta.
giallo zafferrano
Ho tentato di selezionare una foto in particolare aspettandomi di veder
comparire la foto nel dettaglio. Ma ciò non accade. Rimango delusa.

all the cooks
Non le trovo.

\item Avanzare con le fasi di preparazione della ricetta selezionata mediante messaggi vocali
per entrambe
Mi aspetteri un pulsante apposito o qualche indicazione al riguardo. Non
lo trovo e penso che non sia previsto.


\item Text-to-Speech delle fasi di preparazione di una ricetta.
per entrambe
Anche in questo caso mi aspetteri un pulsante apposito o qualche indicazione al riguardo. Non
lo trovo e penso che non sia previsto.

\item Individuare gli ingredienti necessari alla preparazione della
ricetta selezionata.
giallo zafferrano
Immediatamente individuo la lista degli ingredienti.

all the cooks
Immediatamente individuo la lista degli ingredienti.
In più apprezzo molto la possibilità di modificare le quantità degli
ingredienti in baso il mumero di
porzioni selezionato.

\item Individuare la difficoltà di preparazione della ricetta
selezionata.
giallo zafferrano
Immediatamente individuo la difficoltà di preparazione della ricetta
selezionata.

all the cooks
non lo trovo.

\item Inserire una propria ricetta nel catalogo dell'applicazione.
giallo zafferrano
Vado nella home ma non trovo tale opzione.

all the cooks
Vado nella home ma non trovo tale opzione.

\item Condividere una ricetta sui social network.
giallo zafferrano
non trovo il pulsante.

all the cooks
facile. c'è il pulsante apposito.

\item Commentare o lasciare recensioni di una ricetta con la community.
giallo zafferano
Ho trovato il pulsante "scrivi una nota" scambiandola per il tasto
appropriato inquanto non intuitivo.

all the cooks
ho trovato faceilmente l'apposito pulsante.


\item Inserire gli ingredienti nella lista della spesa.
per entrambe
Dalla schermata di consultazione di una ricetta mi aspetterei di trovare
un pulsante aggiungi gli ingredienti alla lista della spesa. Provo a
tornare alla "home" ma non è previsto neanche lì.


\item Accedere alla lista della spesa.
per entrambe
Ho cercato il pulsato apposito nella schermata home ma non l'ho trovato.

\item Rimuovere gli ingredienti dalla lista della spesa.
Fail.

\item Modificare la grandezza del font della ricetta selezionata.
per entrambe
Provo il pinch to zoom e il doppio click ma nulla accade.

\item Creare un menù completo selezionando una lista di ricette.
per entrambe
Dalla schermata di home provo ad effettuare una ricerca con la parola
"menu". Non trovo nulla di appropriato. Inoltre non trovando un pulsante
apposito
nella home credo non sia previsto. 

\item Cambiare la lingua dell'applicazione.
per entrambe
Anche in questo caso cerco un pulsante appropriato nella "home" ma non
lo trovo. Credo per tanto che l'unica lingua prevista sia l'italiano.
