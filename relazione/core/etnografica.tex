\section{Analisi Etnografica}
Un applicazione può rivolgersi a diverse tipologie di utenti.
L'etereogenità di persone che al giorno d'oggi sono tecnologicamente in
grado di utilizzare smartphone e tablet è elevata, anche se va
considerato che esistono molteplici target d'utilizzo che si
differenziano per varie caratteristiche (e.g. fascie d'età, livello
d'istruzione, genere, stato civile, etc.).

Nello specifico, un applicazione di cucina abbraccia sicuramente più di
una singola tipologia d'utente. 
Ad esempio ci si potrebbe aspettare interesse verso tale applicazione
da parte di utenti 
con una predisposizione tecnologia, molto tempo
libero e medio interesse alla cucina, i quali sarebbero tentati di
affacciarsi al mondo culinario, ma anche di utenti poco informatizzati ma la
quale gran passione per la cucina li porta a compiere lo sforzo di
apprendere il funzionamento di software ideato per la
loro passione. \\

Identificare i diversi target d'utilizzo è fondamentale per comprendere
al meglio come strutturare un applicativo software in quanto tipologie
di utenti divere si relazionano con esigenze diverse.

A tal proposito è stata condotta un'analisi etnografica che di seguito vedremo nel dettaglio. 

\subsection{Segmentazione del Target}
Al fine di individuare una categorizzazione di utenti in sottogruppi
omogenei per qualche caratteristica si fa fronte a due tipi di
segmentazione.

\subsubsection{Segmentazione Demografica}
Per comprendere meglio come segmentare l'utenza abbiamo considerato
attuali studi in letteratura che identificano caratteristiche
demografiche utili al nostro caso.
Nello specifico:
\begin{itemize}
\item ...
\item ..
\end{itemize}

\subsubsection{Segmentazione Psicografica}
psico

