\section{Conclusione}
In quest'ultima fase vengono esposte le conclusioni in merito allo studio dell'interfaccia utente progettata per l'applicazione CookApp; successivamente vengono proposti degli sviluppi futuri con la finalità di migliorare ulteriormente le funzionalità già presenti affiancandone di ulteriori, individuando nuovi possibili dispositivi d'uso e migliorando l'usabilità in generale.\\
La progettazione è iniziata focalizzandosi sull'ampia quantità di funzioni necessarie ad una completa e corretta esecuzione delle attività culinarie, cercando di capire quali fossero i problemi delle applicazioni già esistenti. L’applicazione doveva essere un amico fidato in grado di guidare l’utente in ogni step, senza essere nè invadente nè assente.\\
Durante lo studio, si è quindi cercato come strutturare l'interfaccia e i contenuti al fine di accontentare differenti tipologie di utenza, implementando anche funzioni non presenti in commercio e con un occhio rivolto al futuro.\\
L'applicazione finale si configura quindi in una posizione di rilievo, superando, in qualità e usabilità, quelle già presenti negli store analizzati.\\
I test, seppur informali, hanno dimostrato come la nostra interfaccia sia stata più orientata ai task dell'utente, rispetto a quanto visto per GialloZafferano e AllTheCooks.\\
Nel complesso l'esperienza utente è stata positiva, i tester hanno apprezzato l'organizzazione dei contenuti, la facilità nel reperimento delle informazioni richieste e l'interfaccia grafica moderna e completa.\\
Il progetto ha perciò, in questa fase, adempiuto alle aspettative iniziali, riuscendo a garantire un buon bilanciamento di semplicità d’uso e funzionalità avanzate.\\\\

Lo user testing dell'applicazione ha permesso l'individuazione di piccoli errori e mancanze, come la difficoltà nell'interagire con la community tramite richieste di aiuto o la difficoltà a distinguere i preferiti e i menù nel proprio profilo.\\
A seguito della raccolta dei pareri degli utenti, si sono individuati gli errori gravi e sono stati corretti in una revisione delle schermate fallate. 
Dalla valutazione finale si evince che gli utenti sono molto soddisfatti dall’utilizzo del prototipo.\\\\

Gli sviluppi futuri possono essere molteplici e su vari fronti e richiedono un'ulteriore analisi approfondita delle esigenze degli utenti dopo aver usato l'applicazione da noi progettata.\\
Senza dubbio il primo step successivo è la progettazione di un'applicazione per smartphone che permetta di integrare tutte le funzionalità nel mobile tascabile. Auspicabile è la migliore integrazione con gli elettrodomestici smart sempre più presenti nelle cucine.\\
Come fatto notare da utenti esperti, sarebbe utile implementare nuove viste delle preparazione della ricetta che permettano di cucinare più ricette contemporaneamente calcolando come alternare diversi piatti di diverse ricette.\\